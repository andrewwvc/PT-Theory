\documentclass[]{article}
\usepackage[breaklinks]{hyperref}

%opening
\title{A Theory of Perception}
\author{Andrew Van Caem}
\date{13th October 2016}

\begin{document}

\maketitle

\section{Intro}
Here, I'll start off by running through the basics of my theory. I propose that there are specific, discrete kinds of linguistic/musical ability, which can be used to explain why it is possible for human beings to appreciate specific art forms, and that in each individual person, each of these abilities will necessarily be either active or inactive. This places hard limits on what a given person is capable of perceiving and understanding at a given point in time. A person who has not activated an individual ability will simply not be capable of comprehending the forms of art that demand that level of ability, so any piece of music, poetic phrase, passage of prose writing or visual image that demands a specific kind or level of ability in order to be understood will simply not be comprehensible to anyone who has not activated the appropriate level of perception needed.

What are these perceptive abilities then? In short, I believe that I have discovered and can classify two new separate kinds of perceptual ability that previously lacked any theories addressing them. These are 'Encompassing Voice' and 'Super Grammar'. Each one allows the perception of different kinds of structures, and can combine together to allow the perception of specific forms that would not be possible with either alone (for instance, the structure of a symphony).

\section{EVP}

Encompassing Voice Perception (EVP) is, at its core, the perceptive ability that allows you to hear what key a piece of music is in. This does not mean that a person with Encompassing Voice will necessarily be able to name the specific key (or mode), simply that they feel, at any given point, that there might be a certain chord or sequence of notes that should be played in order for the piece to finally feel completed, and that in the absence of this being played, that the music is left feeling tense and unresolved. If you have this faculty active, you should be able to feel what I'm talking about when listening to a piece of music based around this kind of tonal tension. However, it is also possible that you may actually not have this, will hence not feel this at all, and will then likely be confused as to what I'm talking about here.

If you are unsure, then I will try to offer a preliminary method of gauging so. Keep in mind that this will not necessarily be conclusive, as even if you have Encompassing Voice Perception it still takes active concentration on the piece or work in question in order to understand it, and in many cases it may take a number of repeat listenings to take in the form of the work and familiarize yourself with its conventions before you are able to wrap your head around its themes. It is only if you find yourself incapable of following along with melodies in the ways I will detail that you can be reasonably confident you do not yet have EVP. But there is no need to treat this as a test, as something you ought to pass, the point of this is that any understanding you may gain will be yours and for your own purposes.

\url{https://www.youtube.com/watch?v=86WGK9xwjTw}

For example, in the above video (from Stephen Malinowski, whom I owe a great deal of debt to), which I will ask the reader to listen to, a fairly slow paced version of Chopin's opus 27 no.2 is played. At around 0:27, 0:37 and 0:46 certain notes in the main melody will sound. To a non-EVP listener, while the very beginning of the piece may have been somewhat pleasant up to then, they will likely find these dissonances to be grating, and they will unable to gain much, if any, enjoyment or positive emotion out of the following treble melodies as they develop in complexity. At best, they will only be able to really enjoy the bass melody as it repeats throughout the piece. But to a listener with EVP on the other hand, other options are available. They should be able to follow through the dissonant parts, interpreting them instead as a kind of longing for a further response, as opposed to shear dissonance, and feel in the rest of the piece, in an abstract way, possible expressions of compassion, companionship, confessions of worry and all kinds of romantic feeling as the melodic lines passionately 'speak' within themselves and to one another. It is this sense of 'speaking' purely through tone that is my main reason for calling this kind of perception Encompassing Voice. But, regardless of what you are capable of, concerning yourself about how much you like the music or what that might say about you isn't the point here, your taste is your own and this is not something I want people to fret about.

As an additional aside, the main reason I chose the particular particular Chopin piece I did to illustrate Encompassing Voice is that, aside from having the notes nicely illustrated by the imagery in this example, while real appreciation of it does demands Encompassing Voice, it does not really require you to also have Super Grammar, the other ability I want to illuminate. This is not the case in so much other Art Music. Symphonies especially will generally require Super Grammar in order to extend the structure of the musical themes granted by Encompassing Voice and build them up into larger forms. But this is something I intend to address later.


\section{SGP}

Super Grammar/Hierarchical Form Perception (SGP) is more abstract than EVP and better demonstrated through specific examples, but enables 'hierarchical' imagination, the ability to intuitively feel the structure of things like Shakespearean verse and how language can be built up around a beat structure (like poetic metre) so that ideas, phrases and metaphors can be composed in more complex ways. Without it, there will be no appreciation of blank verse or similar forms of poetry (though simpler rhyming poetry will still be accessible), and I will make the claim that it is literally impossible to imagine the metaphorical ideas and images that arise from this kind of poetry without having this mode of perception. Many people have likely had the experience of being confronted with Shakespeare in school and encountering something that seemingly reads like an alien language. For a significant number of them, the idea that this is supposed to be comprehensible English, the same language that they have been using most of their lives, is baffling, even if they simply accept that this is supposedly simply a more ancient form of it. For these people, the best approach to showing their grasp of the material and passing examinations may essentially be to read CliffsNotes, figure out whatever sociopolitical interpretation is most likely to please their teacher, and write out whatever fluff they can manage to in order to give off a convincing impression that they have the ability to interpret the Bard's pen. The person without SGP will have no intuitive sense of poetic metre, and will need iambic pentameter explained to them before they grasp that they have little, if any, feel for it. The person with it, however, should simply be able to read off verse naturally, continuing through line breaks, and find that they can build up a sense of how the structure of the lines is significant and allows them to fit together in order to complete each other, building up ideas and metaphors as they go along.

Now, there may be some confusion here. You see, much old verse, and Shakespeare's especially, does contain a significant number of words and phrases whose specific use and function will not be familiar to the modern reader, and might not even have been immediately comprehensible to even his original audience. This, naturally, is a barrier to understanding, in addition to the extra grammatical demands made by the poetic structure itself. If so, then why choose Shakespeare as the prime candidate to draw initial examples from, as opposed to a more modern poet? The reason for this is that, from the currently limited experience that I can draw on, the density and vividness of Super Grammar based effects and the number and use of hierarchically constructed metaphors employed is greater in Shakespeare than in any other poet I am aware of. It is because of this, I believe, that Shakespeare is so beloved by those who are sensitive to poetry and ideas molded by poetic structure, yet also so despised by those who this is absent in. Without poetic, super grammatical sensibility, which directs imagination and dramatic sense, Elizabethan verse will come off as dry, awkward, confusing and pointless, in addition to feeling like it does not basically follow normal grammatical rules at all. The thing is, it really doesn't (follow ordinary grammar), and this will be a deal breaker for many, but for those who are super grammatically sensitive, the form and structure of the verse is intended to redirect your mental construction of the poetry's form so that otherwise nonsense sentences can instead give rise to novel forms of mental imagery and play.

It is repetition of rhythm, deliberate reuse of consonants (as in alliteration) and in general, setting parralel parts of sentences against each other so that they clash and contrast with each other in order to build up a new idea or feeling, that super grammatical forms give rise to high poetry. The first entrance of the ghost of Hamlet's father is a solid example of the effects that can be achieved with this. In the scene, the ghost appears by breaking the previous line mid way through, creating a sense of mystic horror as the previous flow of the verse is broken out of and responded to via abrupt phrases and hard repetition. This effect and others like it will not be accessible without SGP. There is much more to be said about this, and I intend to provide a larger number of examples, but as I mentioned above, this will be saved for later articles.

\section{Outro}

An important and significant thing to understand about these tiers of perceptual ability is that if you happen to have not yet activated one of them, you can still develop it later. It is absolutely possible for a person lacking a perceptual ability to gain it, and as far as I can tell at almost any arbitrary stage in life. It was experiencing this for myself twice, once for each of the above abilities, that prompted me to start writing this. However, achieving each of them did take a significant amount of work, and transitioning between tiers of ability was very disorienting; the effect of doing it is sudden and demands that you rapidly adjust respectively to each new way of perceiving the world. However, because of each respective transition, and things that were previously impossible subsequently became natural and effortless.

For further reading, I describe what I underwent in the case of each transformation in greater detail in a \href{http://chromaticproduction.blogspot.com/2016/10/five-years-ago-unexpected-thing.html}{subsequent article} (for \href{http://chromaticproduction.blogspot.com/2016/10/five-years-ago-unexpected-thing.htmlhttp://chromaticproduction.blogspot.com/2016/10/five-years-ago-unexpected-thing.html}{EVP} and \href{http://chromaticproduction.blogspot.com/2016/10/five-years-ago-unexpected-thing.html#SGPstory}{SGP} respectively), and provide more basic information about Encompassing Voice \href{http://chromaticproduction.blogspot.com.au/2016/11/encompassing-voice-perception-faq.html}{here}. I also make an attempt to systematically show how these different kinds of perception can synthesize in \href{http://chromaticproduction.blogspot.com.au/2017/05/variant-construction-part-1-definitions.html}{these} \href{http://chromaticproduction.blogspot.com.au/2017/09/defining-poetry-establishing-structure.html}{multi-part} pieces.


\end{document}