\documentclass[]{article}
\usepackage{verse}
\usepackage[breaklinks]{hyperref}

%opening
\title{A Reply to a Reply to Fabb and Halle}
\author{Andrew Van Caem}
\date{13nd September 2019}

\begin{document}


\newcommand{\attrib}[1]{%
\nopagebreak{\raggedleft\footnotesize #1\par}}

In 'Metre is music: a reply to Fabb and Halle' by Bert Vaux and Neil Myler, an attempt to address a particular issue is made, namely that many aspects of poetic meter do not seem to be able to be accounted for if it is assumed that poetic meter, in general, is "projected from the surface syllables of a poetic text" as Fabb and Halle suggest (e.g. as opposed to possibly being projected from things like implied space in between syllables, similar to what occurs to rests occurring in musical meter).

What I suggest instead is the case is something that Vaux and Myler explicitly want to rule out with an appeal to an apparent lack of elegance, namely that, in fact, there do seem to be \textbf{two}, separate different sources of what has come to be understood as poetic meter, and that both may need to be recognised as such in order to account for the full range of metrical phenomena that appear in various corpuses of poetry, and that both overall approaches are essentially \textit{correct}, \textit{complimentary} and \textit{necessary} in order to understand how varieties of poetic structure arise as they do.

Look first at an example Vaux and Myler use to demonstrate the inadequacy of Fabb and Halle's approach.

\renewcommand{\poemtitlefont}{\normalfont\large\itshape\centering}

\settowidth{\versewidth}{The mouse ran up the clock,}
\begin{verse}[\versewidth]

\poemtitle{Sonnet 18}
	\itshape
	Hickory dickory dock,\\*
	The mouse ran up the clock,\\*
	The clock struck one,\\*
	The mouse ran down,\\*
	Hickory dickory dock.\\
	\end{verse}
	\attrib{- Anon}
\section{}

This is a solid example of what I would call a 'sing song' poem. Its beat is the same as that of the essential beat of music, it is \textit{timed}, and due to the meter arising from the timing of the phrases, (as opposed to only surface syllables) certain beats can fall on silences. Assuming its meter arises in the ways Vaux and Myler propose, and that it naturally shares its core beat structure with that of musical beat structure is something I have all reason to agree with. The issue is that I do not believe that this demonstrates that Fabb and Halle's approach is misguided, rather, as I has previously discussed, I believe poetic meters can instead alternately arise from the same source as musical grouping structure, and this type of explanation, which implicitly ranges over syllables which may have more freedom to their timing than musical meter might normally allow, is likely needed to account for the actual structure of much verse poetry.



\end{document}