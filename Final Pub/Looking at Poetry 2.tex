\documentclass[]{article}
\usepackage{verse}
\usepackage[breaklinks]{hyperref}

%opening
\title{Looking at Poetry 2: Sonnet 18}
\author{Andrew Van Caem}
\date{22nd October 2016}

\begin{document}


\newcommand{\attrib}[1]{%
\nopagebreak{\raggedleft\footnotesize #1\par}}
\renewcommand{\poemtitlefont}{\normalfont\large\itshape\centering}

\settowidth{\versewidth}{Rough winds do shake the darling buds of May,}
\begin{verse}[\versewidth]

\poemtitle{Sonnet 18}
	\itshape
	Shall I compare thee to a summer’s day?\\*
	Thou art more lovely and more temperate.\\*
	Rough winds do shake the darling buds of May,\\*
	And summer’s lease hath all too short a date.\\
	Sometime too hot the eye of heaven shines,\\*
	And often is his gold complexion dimmed;\\*
	And every fair from fair sometime declines,\\*
	By chance, or nature’s changing course, untrimmed;\\
	But thy eternal summer shall not fade,\\*
	Nor lose possession of that fair thou ow’st,\\*
	Nor shall death brag thou wand’rest in his shade,\\*
	When in eternal lines to Time thou grow’st.\\
	So long as men can breathe, or eyes can see,\\*
	So long lives this, and this gives life to thee.\\
	\end{verse}
	\attrib{- William Shakespeare}
\section{}

Now, what have before us is one of the most iconic and well know poems in history. But I want to ask you to look upon it with fresh eyes. I want you to forget who wrote it, the subject matter you think it is addressing and all context surrounding why it might have been written.\\

Now, look at the first line:

\begin{verse}[\versewidth]
	\itshape
	Shall I compare thee to a summer’s day?\\
\end{verse}

Who is the addressee of the poem? We can't quite tell yet, we know nothing remember, but I'll suggest something. I believe that the real purpose of the of the poem is to describe who, or what, 'thee' is, and to allow the reader to form a mental impression of their nature. Let's look back at the line again.

\begin{verse}[\versewidth]
	\itshape
	Shall I compare thee to a summer’s day?\\
\end{verse}

Many people, on encountering this will immediately assume something. That is, that the poet is saying "Well, you're clearly so beautiful, so I'll now compare to you to a summer's day and tell you how much more lovely you are". To that I'll simply tell you to look at the second line, and those that follow it.

\begin{verse}[\versewidth]
	\itshape
	Thou art more lovely and more temperate.\\
\end{verse}

Now, what sort of adoration would prompt someone to praise their object of affection by telling them they are more temperate? Milder, less wild, plainer, more demure. It may really be a quality that the poet finds attractive, but it would feel a strange thing to throw out in order to say, woo a lover. So I'll posit something else. In the first line, the poet is honestly asking a question. Let's say that the subject is of such a nature that the poet is perhaps not really sure if they even ought to be comparing them to a summer's day in the first place and so is genuinely asking them, maybe somewhat wryly, if he ought to. So what then is this subject? Could we even say they are human or mortal?

\begin{verse}[\versewidth]
	\itshape
	Rough winds do shake the darling buds of May,\\*
	And summer’s lease hath all too short a date.\\
\end{verse}

The whole of Summer is limited and inconstant; from the start of the Northern May, Summer wrestles the hemisphere from Spring, but then only has control of the season for a time, until its lease ends and gives way to the Autumn.\\

The addressee is \textbf{not} like this.

\begin{verse}[\versewidth]
	\itshape
	Sometime too hot the eye of heaven shines,\\*
	And often is his gold complexion dimmed;\\
\end{verse}

Even heavenly bodies that grant warmth and life are, with regard to their effect on the Earth, things that pass beyond clouds and the horizon. Even the ever-watching face of the celestial will become dark, or overly bright.

This is, again, \textbf{different} from the nature of the true object of the poem.

\begin{verse}[\versewidth]
	\itshape
	And every fair from fair sometime declines,\\*
	By chance, or nature’s changing course, untrimmed;\\
\end{verse}


Still, during the ordinary course of nature, even the merely pleasant and pleasing will be disrupted or fade, whether temporarily or not. And not necessarily even from any special or unusual circumstance that might cut something's lifespan short.\\

Our subject is yet perhaps not even comparable.\\

So up until now, the subject has been described through negatives, though her comparative stability and absence of changing qualities. She is neither like the Summer, which captures a quarter of the year at best, nor the daylight, which still only dominates half the day at most.\\

\begin{verse}[\versewidth]
	\itshape
	But thy eternal summer shall not fade,\\*
	Nor lose possession of that fair thou ow’st,\\*
	Nor shall death brag thou wand’rest in his shade,\\*
	When in eternal lines to Time thou grow’st.\\
\end{verse}

So here we have the second part of the poem, the Eternal, the Eternal Summer. This is permanent, beyond the ordinary summer that effects the globe. Its light is unyielding to death's shadow. Because of this, our subject flourishes as time progresses through eternity, yet is also beyond any transient change that would mar her.

\begin{verse}[\versewidth]
	\itshape
	So long as men can breathe, or eyes can see,\\*
	So long lives this, and this gives life to thee.\\
\end{verse}

So this, this eternal facet will provide the sustenance of the subject (or an aspect of it/her) as long as there are living beings for this to manifest within.\\

So who or what is the subject? Throughout the poem, we are told a lot about them, but only through negative descriptions, comparisons to aspects of nature they are not like, things that have extremes and cycles. We are also told how they relate to 'this', a thing that we are to trust is genuinely eternal. And so the whole of the poem builds up this mysterious image of the subject and hints at many subtitles of their nature. So I will ask you, keeping all this in mind, to read through the poem again. Mind that temperate is supposed to rhyme with short a date. If you have poetic sense, you ought to be capable of, with a little practice, taking in the poem as a whole and understanding how all of it builds up and cumulates into a single idea, the nature of thee, who is maybe incomparable to a summer's day. The exact form of this idea, this person, is not stated, we cannot say that they look like anything at all, but we can use the impression the poem provides us to imagine and grasp at the possibilities of what they could be. Play with these possibilities in your mind.\\

Now, I am aware of the 'traditional' interpretation of this poem in historical context as one of the 'Fair Youth' sonnets. I'd rather put that out of my mind though, as even thinking about that destroys the beauty of the poem as I see it when taken by itself.
\end{document}