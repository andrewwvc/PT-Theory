\documentclass[]{article}
\usepackage[breaklinks]{hyperref}

%opening
\title{My Story}
\author{Andrew Van Caem}
\date{30th October 2016}

\begin{document}
	
\section{}

Five years ago, an unexpected thing happened to me. All of a sudden, the way I experienced sound changed, altering the way I heard almost everything.

It was while listening to this piece (\url{https://www.youtube.com/watch?v=VxzY3tFTz9k}), the harpsichord solo from Bach's 5th Brandenburg concerto played by Gustav Leonhardt, that my transition to Encompassing Voice Perception occurred. It was because of this that I then began to experience higher levels of musical structure. I don't believe that my experience is unique. In fact, I believe I can find examples of people who I believe have been through the same thing, and I suspect that many more may have undergone it without conscious awareness of it happening. The following video provides a few examples, the Matt Giordano and Tony Cicoria sections being particularly relevant.

\url{https://www.youtube.com/watch?v=tqrNEmuSCis}


To give you a more relatable idea of what this general experience was like, there are examples of other people activating a similar kinds of mental transformations due to a deliberate and concentrated effort to change their own mind, such as that of Susan R. Barry, whose case is given here.

\url{https://en.m.wikipedia.org/wiki/Susan_R._Barry}

Here, we have an account of someone who went all her life without having 3D vision, and on top of this being told that after a certain age, it will be impossible to learn if you don't already have it. She ended up showing that this whole point of view was wrong and that transformations of perspective, even literal ones such as this, may be possible at any age.


Another good reference point is this article by Blake Ross.

\url{https://www.facebook.com/notes/blake-ross/aphantasia-how-it-feels-to-be-blind-in-your-mind/10156834777480504/}

If you are unfamiliar with the concept of visual aphantasia (which has only been discussed sporadically until fairly recently), this article should explain what that entails. I have perfectly fine visual imagination myself, but what I really want to talk about is the thought processes of someone who has something like aphantasia, the inability to imagine or perceive things in ways that others can, realizing for the first time that there are people who can actually imagine images in their heads. I can relate to the experience Blake describes here, of assuming that others who have a particular form of imagination are simply using elaborate metaphors, only to later realize they are experiencing something you cannot and that what they are describing is a more or less literal description of their head-space. My own mental transformation allowed me to directly compare what it was like between having not EVP and having it, letting me understand why certain things may feel obvious to some, and nonsense to others.

\section{}

Now, as you might understand, having your perception of reality suddenly change is not something that you happen to expect at any given random point in time, so actually undergoing the transformation was a strange and disorienting experience. Having said that, one of the nicer things about it was that certain somewhat unpleasant sounds, like the screechy, watery qualities of some styles of violin playing, became much more palatable and pleasant to hear*. Significantly, these harsh and unsatisfying qualities seems to be exactly those that transformed into the "voice-like" qualities that drive the emotional core of tonally centered music, the sense that human like qualities can arise from the violin and that these tones essentially speak and be empathized with. To hear 'voice' arise from tones that just before sounded piercing and muddy was the clearest and most striking consequence of the transformation, but other qualities arose at the same time. The next most immediately apparent aspect was the sense of theme that popped up around the same time. Where previously, much of Bach's music sounded like random patterns of notes scattered around in forms that seemed to have no purpose whatsoever, from this chaos a sense that musical structures could respond to each other, parts standing out as echoing one another in sequence, became an inherent part of how I heard the music. Along with this came the sense that these vocal patterns responding to another formed individual 'lines', and that multiple voice lines could be perceived running through the music at the same time. Before this, it was difficult to concentrate on hearing more than one melody, especially if the melodies were very different or out of phase from one another.

But ultimately, all of this was tied together by the realization, once the music had ended, that the piece was 'inside' of a tension that felt like it should resolved by the playing of a particular key at the appropriate time, when the musical theme and the piece as a whole ended together. So at this point I wondered, when people spoke of music being "in a key", are they actually referring to what I had just come to experience, that the entirety of a piece is literally inside this sense of tension, this feeling that the music should progress to a final point, where it finishes on a specific note? Only later did I realize that my hunch was more or less right, that this new experience was exactly what being "in a key" apparently meant and that major discussions of music theory centered around this. And this surprised me because before I thought that the phrase referred to some far more abstract property of music, as opposed to something I could directly and clearly feel but simply hadn't experienced before.

So this was all very unexpected. While I've heard of tone deafness, there seems to be this assumption among musicians that, as long as someone can hear tones and the relations between them, tonality and "the tonal center" are supposed to be something that naturally arise out of this and can be felt by anyone who's willing to listen and concentrate on the music. But through my own experience I can confirm otherwise. The sense of tonal centeredness, the desire that music should structure itself so that it progresses through a 'theme' and returns to the 'tonic' at the appropriate time was simply a completely new aspect to me, one that I simply had no conception of without the accompanying experience of voice-like qualities arising out of instrumental music, along with everything enabled by that.


\section{}


A good part of the reason I believe this event occurred is that I was actually trying to deliberately train my ears through attempts at active listening, with the expectation this this would lead to a deeper understanding of music over time. Before it happened, I had no idea the effect would be so sudden or pronounced, but I was well aware that other people probably perceived things (in terms of hearing and music appreciation) differently to me, and that it was apparently possible through ear training, to develop your internal processing of sound in order to hear deeper levels of subtlety in music. What was surprising was that, instead of progressively learning and enhance my abilities gradually over time as I expected, most of the progress occurred in sudden, highly distinct changes that radically altered my perception of sound and the ways in which I could imagine it. The bulk of my efforts resulted in most of my progress coming in two discreet bursts of change of inner mental activity instead of the relatively long term process of insight into individual pieces that I expected (though this assimilation and progressing understanding of specific works and styles is still important, just no where near as impactful in comparison).


It would be five years until I went thought another change of this kind. During this time, while I had explained to some people what I had experienced, some had responded with curiosity, and others with confusion, and I wasn't quite sure what to make of it myself. The existence of EVP seemed to explain many things, and I've heard a number of reports from others who seem to have transformational experiences similar to it, so I'm confident I'm not some unique basket-case, but I was still plagued by the possibility that there was still something I wasn't getting. I still didn't quite feel that I fully understood what I was 'supposed' to in all the music I listened to, despite the transformation giving me what I thought was full access to this new language of musical possibility. But still, I could not be sure that was everything I had searched for.

It was in going through the second change, the activation of super grammatical perception, that I felt I had clearly found what I was looking for, and what had been missing the whole time. This time, the activation was more subtle than EVP at first, it had no clear instant where it arose and took a day or two to allow me to properly adjust, but ultimately, the effects it had on me were at least as profound and even more far reaching in their consequences. Musical notes markedly felt like they had more impact on attack, and this effect was pronounced on bass tones especially; I could really feel basslines more heavily compared with before was particularly pronounced with the transition to SGP. But otherwise there was not the same radical change of timbre that accompanied EVP. What really cemented the importance of this though, and that it was a significant change like the kind I had experienced previously, what that I could finally, and clearly, feel something that I strongly suspected I was meant to, and which I had actively tried to before, but which no amount of concentration could actually enable. That, specifically, was the feeling that musical themes should 'develop' structurally, and develop in ways that progressively build up multiple layers of tensions in order to lead back to their resolution.

Not, I want to carefully explain what having EVP, but lacking SGP, meant for my appreciation of music. Don't worry if you find the following section confusing, but I simply cannot come up with a simpler way to explain it. While I could follow along with a musical theme, follow the *transformation* of that theme (with some limitations, which I will explain later) and also follow multiple voices with EVP, either with different themes or the same theme in different phases, the way in which I apprehended themes was essentially linear and did not allow (semi-arbitrary) hierarchical transformation. What I mean by this is that, each time I heard a theme I felt it needed to have 'completeness', for each variation of it to go from start to finish without repeating smaller parts of itself (or another theme) before continuing on. Musical themes must follow a strict, relentless continuity in this way in order for a person with EVP, but lacking SGP (which I will term EVA-SGN, for Encompassing Voice Active, Super Grammar Not Active) to feel that they are actually satisfying or make structural sense. A theme must, start, run through its course fully and without  sub repetition of its structure, finish in what feels like the tonic of the key in order to resolve and then potentially pick up again from the start of the theme to do this again. Any music that does not do this either does not demand tonal centeredness to be appreciated (i.e. it has a structure that doesn't require EVP, like pop music) or is not both structurally meaningful and tonally centered with respect to EVA-SGN (i.e. the music relies on hierarchical forms being build up and being nested within one another via a kind of interpreted grouping structure). Because of this, EVP and SGP are both absolutely necessary to allow the full scale understanding of most symphonies and the forms they take.

So, by breaking through my prior lack of SGP, I manage to overcome this limitation. Yet this was only the beginning of understanding what SGP actually seemingly allowed me to do. I quickly discovered that I could comprehend all kinds of poetry with ease that was previously impossible for me to understand. Because of this, I developed a sense of absolute metaphorical imagination, a whole new way of thinking, that I previously did not have access to (the nature of which I will describe in a later article). As such, realized that this development implied far more then I initially thought. I also found that trying to speed read took vastly less effort than it did before. On top of this, feel that the rate at which I can recover from mental exertion also increased greatly, the implications of which could be spectacular, if true. These aspects of SGP are things I wish to expound upon in greater detail.


* I recall reading an article at some point about a music teacher who I believe did independently realize the existence of EVP. He did not call it that or recognize its other properties, but did understand that many of his students found the timbre and texture of an orchestra to be screechy and unpleasant. His solution was to "train their ears" by playing Mozart symphonies, but with the bass notes softened and toned down significantly, so as to encourage them to follow along the major melodies without distraction, the goal being to get them to reach a point where this sense of harshness dissipated, which I hold would have been the activation of Encompassing Voice. Despite my searching though, I have been unable to find any reference for this, so any help here would be greatly appreciated.

\end{document}