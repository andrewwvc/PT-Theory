\documentclass[]{article}
\usepackage[breaklinks]{hyperref}

%opening
\title{Commonality of Imagination}
\author{Andrew Van Caem}
\date{4th November 2016}

\begin{document}

\maketitle

Is it common for people to have mental abilities that for whatever reason simply cannot be accessed, but which lay there in their mind dormant and waiting to be unlocked? I suspect that not only is this so, but that it is likely common for people to be very different in their capacity to perceive and imagine things in particular ways, to the extent that that certain kinds of thought may be impossible unless specific mental connections are made which allow these latent capacities to be unlocked.

Let's assume that people do reprieve the world in radically different ways, and that they also have very different capabilities in their ability to imagine things. There are many good reasons to do so, but the main reason I'm writing this in the first place is that I have personally experienced changes myself which have radically altered my own perception of the world, but naturally I can't expect to my anecdotes alone to convince people, so I will call to my aid the experiences of others, although I don't think it's as unusual as it might seem to suggest that some mental phenomena which might otherwise be commonly considered to be continuously variable actually have discreet phases. An already existing example of something possibly of this kind is perfect pitch, some people have it and can recognize absolute pitches intuitively and independently, while others do not; though in this instance I'm not aware of any piece of music that would actually require it in order to be fully appreciated, and since I don't have perfect pitch myself, I have no way of experiencing what it would be like to. In addition, some people have what might be called visual aphantasia [1], meaning that they have no "mind's eye" or visual plane on which to imagine things, so they could not actually, say, picture a bird in their head if you asked them to. Similarly, many people (mostly women, or more specifically, people with lower levels of androgens [2]) seemingly lack the ability to directly imagine rotatable 3D spaces that aren't currently in front of them. In addition to this, Clive Bell's theory of 'Significant Form' includes the proposition that certain people seem to be able to perceive 'Significant Form' [3] (the referenced article on which also contains a highly relevant section about Bell's relative of musical understanding) and experience its special kind of beauty aesthetically, while others seem to simply not be able to. In this way, visual perception and imagination apparently involves a number of different facets of awareness that may be active in radically different ways in different people. Some people will have some, while others will not. For yet another example, Bob Milne's [4] remarkable abilities may provide an example of a rarer kind of perceptual imagination. I have little doubt that an enormous quantity of argument has stemmed from people possessing different perceptions of reality and yet not being aware of the actual causes of this.

I believe that in addition to the kinds, there are other distinct levels of perceptive and imaginative abilities related to different senses and ways of combining them into ideas and feelings. In various music education circles there are references to certain ideas, that of training the 'inner ear' (i.e. a person's mental processing of sound), of appreciating 'pure musical' form of 'absolute music' and things of that nature, relating to the possibility of developing a person's ability to appreciate more complex combinations of sound than they are currently able to. The various references to phenomena of this nature make it clear that exposing listeners to various kinds of musical material and/or having them undergo forms of musical training in order to attune their ears to more advanced or exotic musical forms is a commonly accepted practice. It is also quite well observed that different cultures and cultural practices often produce very different types of music, which can sometimes be unintelligible to people who haven't grown up exposed to the specific circumstances that allow them to become immersed in a given musical culture.

Yet despite all this, there is a lot to be said about the specific nature of these mental differences that is not discussed and which there seems to be very little awareness of. Proponents of western classical music, for instance, are often perplexed as to why it is the case that their listenership has dwindled over the years leading up to the present and why so many people simply cannot stand listening to 'art music' in the first place. Many of these people also find that conversely, they themselves cannot bear the music that those who don't enjoy what they advocate listen to and find pop music (in a broad sense) to be unpalatable.

[1] https://medicine.exeter.ac.uk/media/universityofexeter/medicalschool/research/neuroscience/docs/theeyesmind/Lives_without_imagery.pdf
[2] http://www.ncbi.nlm.nih.gov/pmc/articles/PMC3270350/
[3] http://www.denisdutton.com/bell.htm
[4] http://www.radiolab.org/story/301427-head-full-symphonies/

\end{document}