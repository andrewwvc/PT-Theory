\documentclass[]{article}
\usepackage{verse}
\usepackage[breaklinks]{hyperref}

%opening
\title{Looking at Poetry 1 In a Station of the Metro}
\author{Andrew Van Caem}
\date{15th October 2016}

\begin{document}


\newcommand{\attrib}[1]{%
\nopagebreak{\raggedleft\footnotesize #1\par}}
\renewcommand{\poemtitlefont}{\normalfont\large\itshape\centering}

\settowidth{\versewidth}{The apparition of these faces in the crowd;}
\begin{verse}[\versewidth]

\poemtitle{In a Station of the Metro}
\itshape
The apparition of these faces in the crowd; \\
Petals on a wet, black bough. \\
\end{verse}
\attrib{- Ezra Pound}
\section{}
This poem is the simplest I know of which I can now enjoy, but which I found off-putting prior to developing poetic sense through Super Grammatical Perception. It has no grand, complex structure that would make it a bother to read to anyone who was not willing or able to put in the effort to grasp it, but nevertheless, I do not believe that anyone who lacks SGP will be able to really feel the effect of it.\\

When you, the reader, first saw this poem at the top of the post, you likely haD a gut reaction on reading it. It may have been positive, or it may have been negative. Accept this feeling, but do not rely too heavily on initial judgement. Try reading the poem again, understanding the context the title gives, and following the rhythm of each line, which contrasts the two images together.\\

We can have the poet himself clarify his intention:\\

\url{http://www.english.illinois.edu/maps/poets/m_r/pound/metro.htm}\\

Maybe you see the station, and feel negative associations because of the wet, blackness of the bough. You might get the impression of a bleak station, slick with rain water, as random faceless people move about in this drudgery, trying to endure the day, while being interchangeable to the person viewing them. That was the impression I first had of it, as something negative, as if it were saying that the human faces, whoever they were, were nobodies on the damp tree branch that was their dark environment. There was nothing I could do but see this as a kind of bleak portrait of humanity.\\

But maybe this is not the case for you. Maybe you are capable of taking on the whole poem at once, and in a way that bypasses this natural sense of bleakness, seeing it purely as a novel and striking contrast of two separate images, with no intention of saying that the qualities of one are like the other in a negative way.\\

In this case I will make the claim that you likely have the poetic sense that I call SGP, which allows you to intuitively feel contrasting images build up into abstract impressions.

\end{document}