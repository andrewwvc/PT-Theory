\documentclass[]{article}
\usepackage[utf8]{inputenc}
\usepackage[breaklinks]{hyperref}

%opening
\title{Defining Poetry - Differentiating: Part 1}
\author{Andrew Van Caem}
%\date{}

\begin{document}

\maketitle

\section{}

This article needs a particular preface. I did not intend it to become as long as it did, and nor did I start out intending to write the conclusions I actually ended up reaching, as the process I went through to reach them was grueling and what I ultimately discovered surprised me and forced me to throw away much of what I had initially assumed I would be able to say. I started out with the intention of showing that my conception of Super Grammar could be reconciled with other approaches that grapple with the peculiar grammatical properties observed in poetry, assuming that comprehension of these grammatical properties would be largely dependent on SGP and that things like 'displacement' (the use of words out of their normal grammatical order) would be comprehensible in specific ways only to those who possess this particular faculty. I ultimately had to revise this into an understanding that the 'ungrammatical' 'free order' of words in poetry (as compared to 'ordinary sentences'), and the natural ability for people to comprehend this, despite finding such speech odd or even nonsensical in other contexts, is something that seems to be largely independent from SGP. The ability of those who are capable of using the SGP/VI to gain unified vitalistic understanding of Ideas from language presented in ways that emphasizes parallels between parts of it is seemingly not necessary to account for the innate ability to comprehend 'poetic grammar', on a basic level. In other words, the ability to make sense of 'ungrammatical' speech, due to words being in the 'wrong' order, even in poetry, seems to be something that most people are innately capable of, regardless of whether or not they possess SGP, though SGP seems to be able to aid comprehension of certain constructs in specific circumstances and may enhance that 'naturalness' of elaborate uses of grammatical displacement. Given I came to believe this, as opposed to what I initially hypothesized, what then seemed to cause this confusion initially? In part, gaining SGP did actually seem to affect the way I perceived poetic displacement and free ordering, but it turned out that these changes could be accounted for via what I've termed 'multiplicity', [link appropriate place in Poetic Imagination article back to this] which allows a given instance of a word to be more easily interpreted in distinct ways, which naturally seems to interact with displacement.\\



Ultimately, my aim for this article is to show how it should be possible to distinguish between 'high' or 'literary' works which make significant use of the faculty of Vitalistic Imagination and how this might help to solve certain long standing issues in the aesthetics of literature. While I assumed that out-of-order displacement of words would itself be a major part of this, this seemingly turned out to be an unnecessary conjecture, but I will leave the examination of this in as part of this work, because it has significance in its own right and is necessary to demonstrate certain issues.\\



\subsection{}



To begin, I want to introduce a new term. I have previously brought up both Supper Grammatical Perception (SGP) and Vitalistic Imagination (VI) as individual phenomena and yet implied that they are strongly and inseparably linked, really being manifestations of a single underlying complex. So, to create a single term through which I can properly treat and refer to them as aspects of the same single faculty, I hence call this the Super Eidosic Complex (SEC), within which SGP and VI are unified into a single modal faculty. [SEC]



\subsection{}



In this article, I want to tie what I have been exploring into other theories and accounts of poetic experience. To do this, I will both detail more particular qualities of poetic structure and how they arise, and bring in specific theoretical approaches and try to show how my ideas might integrate into them, in ways that are both complimentary to and critical of them. So I will focus in doing the following, detailing two separate structural qualities that I want to bring to attention, which I call multiplicity and out-of-order interpretation, and showing how they (along with the existing aspects of poetry I have already identified in prior articles) are enhanced by the possession of the SEC and how they fit into and augment other contemporary approaches to poetics. Specifically, the other approaches I want to bring up are:



Poetic language: a Minimalist theory, a PHD thesis authored by Gary Stewart Thoms

Aristotle's Poetics

Selections from Owen Barfield's articles

A theoretical approach to poetry based on various statements made by Roman Jakobson

The Life and Death of a Metaphor, by Josef Stern

Nigel Fabb's [INSERT ARTICLES]



This article will detail particular qualities of poetic structure and how they arise. In order to do this I've realized that I will need to specifically describe some of the structures used in Poetic Language: A Minimalist Theory in order to properly explore the implications of them on particular phenomena. I did not originally intend to do this, but certain concepts I wanted to explore seemingly turned out to possess more subtly and complexity than I initially suspected, and this will require me to import a more comprehensive number of structural aspects from the thesis in order to properly deal with them. Now, I must preface this by saying that, while the specific details I discuss are important, in that their specificity is necessary in order to demark the nature of the structures I wish to discuss, I don't want to give the impression that a person who is actually experiencing poetry, and who has the imaginative capacity to do so and pick out its richness of meaning as it ripens in their mind, is necessarily limited to the specific structures that I discuss here in their strictness. I suspect that the actual totality of the structural relations that occur within poetry that can be felt to be meaningful is vaster and frequently dependent on specific subtleties of feelings that won't be easy to generalize formally. I posit such structures in order to make it clear the freedom that such imagination enables, and to illustrate why it may be difficult for those who lack this imaginative freedom to feel out this potential richness in the same way. I don't want to imply that the meaning someone can gain from poetry is only possible through the transformation grammars discussed, rather they are a tool I make use of to demonstrate the potential difficulty of obtaining this meaning without SEC (and hence without one's mind being aided by capacities and tendencies to build up specific mental forms).



Poetic Language: A Minimalist Theory proposes the idea that poetic language, as defined within it, can be said to differ from what might be labeled 'ordinary language' by diverging from its grammar in particular ways. Essentially, by using the occurrence of certain features in the corpus of works deemed poetic that he selects [SELECTION], he is able to illustrate certain kinds of deviations from 'ordinary' dialects of language that typically occur in poetic language. While the thesis itself naturally goes into all kinds of in depth beyond the sketch of it I give here, and provides much greater detail regarding the detailed reasoning it gives in justifications of its arguments (crucially including the fact that his approach denies the needs for whole new set of grammatical rules and justifies itself by a particular approach to the 'economy' of poetic derivations), I mainly want to mine it for particular classifications of poetic 'deviance' along with other ideas. As such want to quickly provide a short overview of some of the distinctions made...



Deletion: General means of talking about removals of words from sentences that "should be there" in order to make sense of things that are not fully stated. Covers both normally grammatical removals (Ellipsis, covered below) and 'unlicenced' ones (Erasure, below that) in general.



Ellipsis: Removal of certain implied words from a sentence such that it will still make sense as a grammatical sequence. \url{https://en.wikipedia.org/wiki/Ellipsis_(linguistics)} Covers such example as —

"John can play the guitar, and Mary [can play] the violin."

where the [can play] in square brackets can be omitted, thought its meaning is implied and it could optionally be included in the sentence in full. Though here I caution against committing to the simple assumption that any sentence which could modeled as a longer one with certain words omitted (due to ellipsis) is 'merely' equivalent to that longer one except having gone through some particular formal transformation, as opposed to potentially "standing by itself" and/or having unique connotations to its meaning. [MOVEMENT GRAMMAR]



Erasure: Specifically 'poetic' deletion of words for the sake of poetic effect. Specifically introduced in The Minimalist Theory. The following is an example given:



...it is in the disrepair

Of these lives that we     not find despair



Where the blue mark is the place where, it is claimed, that the appropriate implied word is erased from (though I don't necessarily agree that this has to be the case, or that the mind necessarily comprehends sentences like this by always "filling in the blanks"). The Minimalist Theory only considers certain forms of this to count as erasure, and requires that they be transformations of actual existing sentences, but still permits this in ways that don't produce normally grammatical English sentences. I want to point out here that I don't necessarily consider erasure, as proposed in the thesis, to be something distinctly enabled by Super Grammar, as I don't necessarily find that the specific examples The Minimalist Theory gives in order to establish its definition are things that I find to be significantly more comprehensible after the obtainment of SGP, compared to before. I also don't feel that the Beckett examples in particular strike my Vitalistic Imagination in any particularly vivid way that would be an indication of the use of the SEC. So even if this is ultimately relevant to the theory of possible poetic freedom that Thoms proposes, and aspects of this turn out to be validated by further evidence, I will consider this part of the theory to be of relatively minor importance to what I am specifically interested in exploring here compared to other aspects.



Movement: A concept  (in transformational theories of syntax) which in part deals with the possibility of words in a given sentence having parts of itself rearranged (and hence, moved from one place to another) in order to form another sentence with a similar (possibly even 'the same') or related meaning. Regular, non-poetic language seemingly allows forms of movement that are considered both meaningful and grammatical and which seem to explain why certain forms of sentences seem to be 'morphs' of other related sentences in structure, in terms of having different word orders while carrying their parts of speech in different ways. This is a phenomena in generative grammar with a great deal of scope. url{https://en.wikipedia.org/wiki/Syntactic_movement}



On the stage stood an actress.

An actress stood on the stage.


This is an example of a form of movement that is termed 'locative inversion' (which in this case, does preserve meaning, though other forms of movement can enable sentences with distinctly different meanings, such as wh-movement). [MOVEMENT GRAMMAR]



Displacement: The phenomena related to movement (according to A Minimalist Theory) that occurs in language, which allows parts of a sentence to move from their 'normal' positions in 'ordinary' language to 'unorthodox' positions and yet still be interpreted as being meaningful, despite not being explained by any normally acceptable movement operation. Poetry seems to permit forms of rearrangement that do not occur in what are seen as regular dialects of a language, and as such there exists specifically poetic forms of rearrangement that The Minimalist Theory seeks to explain, which are termed displacement. Here is an example of such a displacement (with the displaced part underlined and the place where it seemingly 'ought to be' marked in blue):



For lo! the board with cups and spoons is crown’d    ,

The berries crackle, and the mill turns round;





Poetic language seems to allow normally 'non-grammatical' displacements that still seem to be interpretable as ordinary sentences.



\section{}



I mainly bring up Poetic language: a Minimalist theory because I want to bring in something relatively recent which systematically grapples with with poetry as a linguistic construct and, at the same time, addresses and criticizes a broad swathe of the alternatives that exist in the current literature, and which as such attempts to create a coherent theory by addressing perceived issues with those other approaches. But first I want to draw attention to a particular claim made in its conclusion which I specifically disagree with and which I believe can be attacked in order to support the views of mine which strongly diverge from the thesis. An chief assumption made in the thesis, which is directly opposed to my own experiences, is that ultimately, the meaning of a given poetic sentence can be said to be directly equivalent to that of an ordinary sentence, i.e. there is a sentence to which no specifically poetic alterations from the 'normal' grammar have been made, which that poetic sentence can be considered a transformation of, and which ultimately means the same thing. Now, this is a greatly simplified paraphrase from what Gary Stewart Thoms has actually claimed, especially considering that his approach to modeling grammar in order to account for 'poetic language' is very different from mine and its creation and development is motivated by different reasons. However, I draw the comparison to explore how an alternate attempt at creating a 'poetic grammar' can differ when drawn from different assumptions and justifications, and to show that the systematic undertaking that it is, as an attempt to put other existing poetic theories into perspective and expose their potential flaws, doesn't show that the ideas I propose are wrong. The Minimalist Theory makes particular assumptions about meaning and its relation to structure, and because of this it has no reason or motivation to take into account the structures that are of prime importance to my own theory. It is not necessarily flawed on a fundamental level as its aims differ from mine, and I am not bringing it up to simply disparage it and the content it does deal with, as it even accounts for the possibility that there may be aspects of poetic meaning and grammar that simply happen to be outside the scope of its approach, but its core assumptions are nevertheless irreconcilable with my experience of poetry[HEGEL] and so I must point out why any potential criticism based on it is not justified on the basis it provides. The thesis also assumes that the metaphors of poetic language are the same in kind as that of ordinary language, only except they often happen to be more complex and elaborately developed. This I also deny—my experience is that even relatively simple metaphors can be specifically poetic in a way that will be inaccessible to those who lack the faculties to perceive them, and hence different in kind (pointed out here http://chromaticproduction.blogspot.com.au/2017/08/poetic-imagination.html ).



Whereas the minimalist theory justifies things with respect to the existing linguistic theory dealing with ordinary language (exemplified by Chomsky's Minimalist Program, which it was created in line with) and the tools it offers for grappling with linguistic structures, I have already offered reasons to approach poetic language as an relatively exceptional phenomena in ways beyond this, namely that SGP seems to offer the potential for a given mind to perceive meaning via specific kinds of structures which are not accessible to minds without it. As such it is reasonable to develop a framework which stands at least partly outside of the scope of the specific methods used by Poetic language: a Minimalist theory. Part of this should involve an attempt to explain what kinds of 'meanings' (in a very general sense, that may include things which might not be considered part of the meaning of ordinary language in a stricter sense) the experience of poetic language offers that don't normally play a part in non-poetic language.



In my previous article ( http://chromaticproduction.blogspot.com.au/2017/08/poetic-imagination.html [link appropriate place in Poetic Imagination article back to this]), I offered something that attempts to demonstrate this. In particular, I wrote this

"It is in the composition of multi-layered metaphor into particular kinds of unified, vitalistic and abstract-yet-vividly-felt Ideas that VI (Vitalistic Imagination) is uniquely distinguished." I did not elaborate much on what I meant, but I want to draw attention to this statement and to use examples of poetic works I've already discussed and analysed before in order to make it clearer.



Again I quote Proust's In A Station At The Metro:



The apparition of these faces in the crowd;

Petals on a wet, black bough.



I have experienced this poem in two distinct ways. One, while lacking in SGP, and the other while having it. Because of this, I can say exactly (as far as I can recall and understand my own experience) what this poem offers to the poetically sensitive person that will not be available otherwise and why this is specifically dependent on the SEC, and hence unique to 'poetic language'.



What mainly stands out is the way the juxtaposition of the two lines can be experienced. With the SEC, the reader is allowed a kind of free play of (more than 'merely' [link this]) conceptual abstraction that arises from these two ideas being expressed directly next to each other in the particular way that they are. Namely, while the two phrases/ideas are contrasted, they are not necessarily forced into any particular, specified relation. You are not told that the "apparition of faces in the crowd" is like "petals on a wet, black bough", nor that they are unlike, or similar in appearance in a specific way.* Instead you are struck with a kind of brilliant impression of the relation, as expressed by the poem—in its totality. While I won't claim that this meaning/experience is necessary outright impossible to express/induce via other means, it seems to me unlikely that things that are not the poem itself will be able to achieve this.

Insert comment on Wittgenstein's thoughts on aesthetic irreducability.



I will not claim that this relation is necessarily completely abstract, but it is in some fundamental way freer than the relations that a non-poetic interpretation of the lines would naturally allow. As such I believe that specifically poetic expressions can provide meaning/experience that the experience of 'equivalent' ordinary language doesn't provide, and that much of this will only be accessible to those who have obtained the SEC.** What I do want to show with it though, is that, regardless of concerns of meter, word order, multiplicity, erasure and complexity of structure and metaphor, this act of basic, direct juxtaposition can give rise to specifically poetic experience, and as such, this example serves as a basic demonstration of what is poetically irreducible to ordinary language, even when the complexities of "specifically poetic" grammar are not a particularly deep concern here.***



\subsubsection{}



*Pound does comment, in an article he wrote about the composition of the poem, about his translation/paraphrase of another poem, a haiku (hokku) which goes like this:



"The footsteps of the cat upon the snow: 

(are like) plum-blossoms." 

In this instance, Pound explicitly adds  the words (are like) for 'clarity'. This suggests that, while it doesn't need to be said, or isn't explicitly said, likeness is one of the predominate relations that naturally occurs one reading poems like this.



**An SGPN, however, may be reading the poem and still wonder what it is that their experience is actually missing despite this explanation. After all, they are, as it seems, perfectly capable of looking at these two lines and thinking/imagining contrasts between them. If there is some experience that they do not have, and there is supposedly also nothing else that they have experienced so far that would be comparable to it, then how can they be sure they do actually lack it, given they may simply not be fond of the poem even if they do have SGP? In this case, the example here is used due to its simplicity in illustrating the most basic and seemingly fundamental elements of high poetry most clearly to those who do possess this capacity, and not because I thought it would by the best example for the sake of constructing a test, so an individual liking or disliking this poem may not be the best indicator of whether or not they posses SGP. If someone is actually interested in testing themselves, to determine if they possess SGP, then it's best that several other examples with more elaborate constructs, and which include multiple forms of art, should be used for this sake.



***Though how the two lines specifically match up via the mental comparison between their beat structure that naturally arises in the poetically sensitive person is still somewhat interesting and has its own subtleties. In my impression: 'of' matches with 'on', 'a' / 'these', 'wet' / 'faces', 'black' / 'in the', 'bough' / 'crowd'. 'The apparition' seems to ambiguously overextend across 'Petals' though, which may be an essential part of the experience, so I won't attempt to define a comparative 'scan' for it. So, in this case, despite the relatively 'freeform' nature of the poem, SGP and the kinds structure it enables still play a crucial role in its experience.



\section{}



Multiplicity and Out-of-Order Interpretations:


Multiplicity of interpretation and Out-of-Order interpretation are two different but related aspects of what could be termed poetic language that differ from what might be called 'non-poetic' language. Multiplicity is a specific dynamic sense of experiencing ways that an expression may take on multiple meanings, and is enabled by SGP. I have made mention of it in a prior article. Out-of-order interpretation is mostly an extension of displacement as discussed above, though with the added acknowledgement that it may alter the sense of the interpretation in ways that are affective (i.e. involving feelings instead of predicative 'facts' of its meaning). Otherwise, Out-of-Order interpretation is mostly interchangeable with displacement. I do not at all claim that either of these phenomena are, in their broadest sense (which admits comprehension of meaning in terms of 'merely conceptual' understanding), impossible to comprehend without SGP/VI, but I find that there are many cases where SGP biases my sense of interpretation in ways that make instances of poetic language where these occur much more apparent and natural to perceive, instead of feeling barely apparent or forced (as per when I was SGPN), and increases the range of their expressive possibilities by allowing vitalistic understanding of them. The real reasons I bring both of these up together, as you will see, is that it will allow me to pinpoint exactly what I believe the SEC grants me, which involves the combination of the two working together in a specific way, which allows vitalistic thought to wrap around displaced word orders in a unique way in order to build up distinct multiplicities of meaning. This is what I initially assumed was evidence for the idea that the SEC actually enables distinct sentence structures based around unique kinds of displacement, which I ultimately rejected. That being said, I don't reject the idea that Super Grammar allows certain forms of grammatical structure, build around variant addition, to create long form sentences with vitalistic internal parallels that won't be felt to be meaningful in its absence.



Multiplicity of interpretation is the possibility of having a single expression mean two things at once and have the subject/understander feel that the expression does actually mean both of these things simultaneously, though in a particularly fluid way that is distinct to vitalistic thought. The enhancement of this mental capacity could perhaps be more accurately be considered an aspect of VI than SGP (even though both SGP and VI are effectively labels for different aspects of the same underlying thing, in the SEC), but it can be seen as an aspect of variant grammar all the same (as it can be brought on through parallel structures and decoration biases).



For an example of this that offers some meaningful difficulties, look at the following exchange from Shakespeare's King Lear [CHANGE NOTE]:



KENT

Is not this your son, my lord?

GLOUCESTER

His breeding, sir, hath been at my charge: I have
so often blushed to acknowledge him, that now I am
brazed to it.

KENT

I cannot conceive you.

GLOUCESTER

Sir, this young fellow's mother could: whereupon
she grew round-wombed, and had, indeed, sir, a son
for her cradle ere she had a husband for her bed. 

Do you smell a fault?  

KENT

I cannot wish the fault undone, the issue of it
being so proper.



You may be able to see here that the word conceive is used in two different senses. First, by Kent, to say "I do not understand the meaning of/reason behind what you have said." and then, by Gloucester, to say "I was able to impregnate (the mother of my son)". But one aspect of this exchange that should be apparent is that it isn't two instances of the word conceive being used, with each instance of the word being used in the same way, but rather a single instance, which Gloucester carries on with a sentence with an implied meaning: "Sir, this young fellow's mother could [conceive (my son) (through) -me-]", with the section in square brackets being naturally unsaid implication (which even then is only apparent after Glouster speaks of her 'round-wombed-ness') and the -me- implied through a transformation of person from the 'you' of the previous utterance (which is itself used metonymically, in place of "of what you mean").[FN1] Now we could also say that the square bracketed part only needs to be implied in its meaning, and doesn't need to be subject to this kind of formal transformation—which is perfectly reasonable[EM]—but the fact that the single instance of conceive was already used in the context it was and naturally ends up inspiring the subject to feel this particular implication in context (at least if they possess SGP, without which this is more difficult to perceive) is something that I have no better means of explaining, as it seems to take advantage of structural similarities between the expressions to craft it (the intuition). The combination of all these different transformations, along with the general unfamiliarity of the conventions of Elizabethan language in general (which may cause the audience to suspect that a given line may be uninterpretable, not because they haven't thought carefully enough about it, but because they are not privy to some context, or specific aspect of the language or culture of the time that they might need knowledge of), seems to make what might be otherwise straightforward wordplay into something that is less natural to interpret than Shakespeare may have intended. I will claim, based on my own experience, that attaining SGP does make this dialog significantly easier to perceive as something that, while not being natural speech, at least something that feels like a natural and sensible exchange in context. Whether this feeling of unnaturalness of the production of meaning here is caused chiefly by the multiplicity of interpretation, or whether it is a combination of this and other transformations and omissions that occur in the exchange that actually to make it more difficult for an SGPN to interpret,  I cannot say, but I still hold that this is an example of specifically poetic language, the full experience of which seems to require SGP even if some aspect of its meaning could otherwise be explained through conceptual paraphrase. A major consequence of possessing the SEP is the sense that both senses of 'conception' can become linked in the mind of the subject, dynamically forming a unified concept in which 'conception' (in the sense of understanding) can vitalistically take on qualities of 'conception' (as the physical act of creating life) and each meaning can be seen as branches of something greater.


Having said all this, I am hesitant to claim that all sentences, whether they be 'poetic' or not, really have a 'meaning' (which I consider to be a totality that might contain both propositional and affective content) that ultimately corresponds merely to some underlying base structure that they are a transformation of [STRUCTURE]. Rather, as I have hinted at before, I do believe that differences in structure themselves contribute to different forms of expression, including expressions that may not otherwise be possible to easily convey in another form.



DISCUSS THE INTERPRETATION OF 'ISSUE' FROM THE SAME EXCHANGE.



[CHANGE NOTE] I initially intended the discussion of this example to be much simpler and more straightforward. However, on actually grappling with it and trying to understand why it seemed so initially non-obvious, and attempting to fit this explanation in with existing grammatical frameworks, I found that the number of issues I deemed necessary to properly deal with it exploded, and so it failed to be the basic illustration of the issue that I hoped. However, I've decided not to throw it away or replace it with something else, as it is this very complexity, and hence difficulty in slotting it into a given system in order to describe the nature of the problem, that seems to be the root of the challenge in naturally interpreting it. As such, rather than showing that 'multiplicity' in a simple sense is something that in itself poses an inherent difficulty that (at least in this instance) requires SGP to be overcome, I pose a distinct example of something which, in its actual complexity—which includes multiplicity as a part of it—does (based on personal experience) seem to demand SGP in order for its 'intended' interpretation to naturally (at least in the sense of feeling poetically natural) arise in the mind of the hearer. This hence provides a concrete example of the kind of interpretive difficulty that this faculty can aid the overcoming of (if it does turn out to be the case that other SGPNs have similar difficulty intuitively grasping it that SGPAs in general do not). At least, the main difficulty here seems to be in taking part of one utterance, interpreting it in one sense in one context, and implicitly (though whatever mental mechanism) transferring this partial utterance into a second context ,where it has a different meaning, on realizing that it makes sense to do so, and feeling that both utterances form a unity. And this is another issue in itself, as the hearer might be tempted, on an initial hearing, to interpret everything uttered by Glouster after 'could:' as a parenthetical aside prior to the compliment that would combine with 'could' to create a complete phrase, which ultimately never comes. This implies that they would need to realize, at least after the fact, that this is the case and to gain from this understanding a drive to reinterpret the previous sentence, unless the actor's intonation made this more apparent. This is not impossible for someone lacking SGP, but I contend it would be much more difficult to naturally interpret such language as making 'poetic sense' on the fly, and in such a way that each sense of interpretation feels both distinct, and yet also unified in their apprehension, as opposed to being understood through more analytical means. You see, the potential vitalistic unity here implies this, that the two apparently (possibly?) distinct meanings are felt to be merely points or moments on a possible spectrum of meaning that the word covers, regardless of whether or not anyone has used meanings that might correspond to points 'in-between' the two meanings prior to this particular expression. This ease of perceiving meaning to be something malleable and flexibly abstract in this way is what I hold to be an element that characterizes vitalistic thought, and seems to make these kinds of expressions more natural to comprehend.



[FN1] A 'direct' transfer of implication from the first sentence to the second would be:

(I)/(This young fellow's mother) (cannot)/(could) conceive you.

This shows the inadequacy of directly importing the implication as a simple transformation, the sentence suggested makes no sense in itself and has to be read into. Even taking 'you' as implying 'me', more words are left unsaid, and it is up to the subject to pick up the difference. This would imply that either erasure phenomena are implicitly involved, that some additional transformation occurs (that implicitly both converts 'you' to 'me' and takes the 'me' as metonymically implying his own son) or that the spectator simply feels out the implication without needing there to be an explicit transformation to some more fundamental structure.[EM] It might also be reasonable to interpret the phrase as:

This young fellow's mother could [conceive].

With the 'you' being dropped and the more ambiguous meaning being initially implied, but this dropping seems to intuitively create more difficulties than it solves.

At the very least, if we assume that the meaning of the sentence is gained from some structural transformation, then this transformation cannot simply, add "conceive you" to the end of the second sentence, and nor can it be taken in the same sense (which would be normal in implicitly elided phrases) which seems to imply that more cognitive work has to be done.

[Discuss the difference between ellipsis and erasure phenomena and how they are both implicitly involved in this example]



[EM] There are many proposals that treat elided phrases 'pragmatically' in this way. 'Simpler Syntax', by Ray Jackendoff and Peter W. Culicover covers some arguments for this.





\section{}





Out-of-Order interpretation is when the 'normal' grammatical order of words in a sentence is broken and rearranged into something that seems to imply a similar meaning (though possibly with difference nuances and structural features), but which is perceived as something that is not grammatically correct in ordinary language. The ability to do this in many particular ways seems to be something that most people can do without needing to possess SGP, the way most people can comprehend Yoda's speech in Star Wars and many other variations of word order in different dialects should attest to that. However the way this is often done in way that seem relatively free and irregular (as opposed to consistently using a characteristic set of grammatical features in a dialect) in poetic writing can throw up problems regarding comprehension of the form and meaning of the writing, which I personally attest to.



The issue of ungrammatical ordering of poetic writing is something that a number of linguists have grappled with. The issue is discussed as a kind of displacement phenomena in Gary Stewart Thoms' PhD thesis [1], but I want to analyse aspects of it here, such as the ways it is used specifically in poetry uniquely, so as to show how apprehension of it and the total effect it may have been intended to have might in part depend on SGP. In particular, it seems to have a special interaction with multiplicity, which biases a reader towards interpreting words in multiple orders and considering the potentially different meanings that each of those orders might provide, and this in particular seems to be inaccessible to those without SGP. This I will go over shortly. But first I want to start off with more basic examples of out-of-order phenomena in poetry:





Her lively looks a sprightly mind disclose,

Quick as her eyes, and as unfix’d as those;

-Excerpt from Alexander Pope’s ‘The Rape Of The Lock’



-Comment here about intra-line the parallel grouping suggested by the conjunction between Her lively looks and  a sprightly mind, which wouldn't be as strong in the absence of the 'disallowed' movement. Though the rhyming is an important feature too.



Here is an example of such an occurrence. We can feel that while, there is some sense to be made here, if something like this were to be said allowed in a clearly deliberate way, by someone speaking in a casual social context, that it would feel 'off' and queer, as it is outside the ordinarily acceptable grammar that we are likely used to. This is not to say that normal speech will always follow a given grammar and that anything said that doesn't conform to it would be felt to be similarly strange though, people slur, repeat words, inserts pauses and 'empty speech' into expressions, create idioms and conventions on a whim that don't conform to the formality of a given grammar etc. And yet, there does seem to be something different about poetic expressions like this, in that they seemingly alter the way we generalize grammar deliberately, in ways that wouldn't come about in ordinary speech. It doesn't seem to be the case the poets came together and decided to create a poetic dialect in which this kind of writing is supposed to work by convention, rather it makes more sense to see that it might be natural, that in composing such verse, the writing here was intended to be felt as a variation of a more normal dialect, as opposed to being an independent dialect in itself that might be used in a social context.



We could view this as a displacement phenomena, in accordance with a Minimalist theory, and I will not deny the use of this kind of analysis. Yet, at the same time, to view such deviation merely as deviation from a standard, where the 'real meaning' is to be discovered by finding the specific standard that is deviated from, and not see the rearrangement as a distinct form of expression itself, is a harsh mistake.



In Owen Barfield's article on 'Form in Poetry', he has this to say about poetic rhythm and the ability of poetry to licence the movement of words from their normal positions: 'It is a beauty fashioned by instinct out of alliteration, assonance, and all the varying cadences that arise from the delicate superimposing of the natural speech-rhythm on a regular verse-rhythm. This music is quite inseparable from what I have described as “form,” for an unmusical phrase can never even “mean” the same as a musical one. The finer and more elusive of the word-associations are not elicited and blended in the same way, so that the state of memory produced by one is different from that produced by the other. Moreover, phrases of a different music must convey a different meaning. For example, an epithet following its noun does not mean quite the same as it does in its normal position in front.'



As examples, he gives quotations of lines from Milton's works:

"Meadows trim with daisies pied"



"Teiresias and Phineus, prophets old"



It certainly seems meaningful to say that there is a sense here, in which "Trim meadows with pied daisies" and "old prophets"* don't have the same meaning—the latter feeling incredibly dreary and almost insulting without the nobility imparted by the poetic phrasing (and SGPNs should feel the same way about this in this particular instance too).



\subsubsection{}



*The word order in the second phrase here though seems to possibly imply something closer to "prophets of old", though not strictly so. I will leave this suggestion, with this potential alternative explanation involving some form of erasure being an ambiguous possibility, with the reader.



\section{}



Aristotle has also commented on this phenomena in his Poetics (in sections 21 and 22 in particular), and spoke of how how poetic diction, in its choice of words, order, presentation, and use of metaphor, benefits from being written in a way that feels foreign, like an exotic tongue that is still intelligible. As such, it may make sense then, if we are to consider poetic writing as a form of dialect, to see it as a kind of meta-dialect, one which has been taken up by poets in various ways in many languages for the qualities it imparts on their works, rather than something which has accumulated via cultural accident. You can see that, even when speaking of the Greek of his time, he calls attention to how the language used in poetic speech will specifically make use of 'unorthodox' word order in even commonplace phrases: "domaton apo, 'from the house away,' instead of apo domaton, 'away from the house' ".





So we seem to have, in poetic diction, a whole offshoot of normal language, one that seems to preserve certain qualities even when the source language from which the poetic language differs.*



-Possibly use hints from the critics and audiences Aristotle mentions that suggest that not everyone was appreciative of the language used, and that as such perceptions may have differed that caused this disparity.



\section{}



Finally, here I want to look as something that might be considered a combination of multiplicity and displacement.



Batter my heart, three-person'd God, for you 
As yet but knock, breathe, shine, and seek to mend; 
That I may rise and stand, o'erthrow me, and bend 
Your force to break, blow, burn, and make me new. 
I, like an usurp'd town to another due, 
Labor to admit you, but oh, to no end; 
Reason, your viceroy in me, me should defend, 
But is captiv'd, and proves weak or untrue. 
Yet dearly I love you, and would be lov'd fain, 
But am betroth'd unto your enemy; 
Divorce me, untie or break that knot again, 
Take me to you, imprison me, for I, 
Except you enthrall me, never shall be free, 
Nor ever chaste, except you ravish me.



This is a prime example of something which I personally considered to be strong evidence of my initial hypothesis, of the ability of SGP to enable displacement. But now, it at least seems clear to me that while SGP is involved in the my experience and interpretation of a specific phrase, it is not in itself what makes displacement in general possible. We will focus on the lines:



Reason, your viceroy in me, me should defend,

But is captiv'd, and proves weak or untrue.


This ought to strike anyone reading it as a very unusual phrase, even for poetry. Its meaning, especially on a first reading, is full of strangeness and ambiguity, and how it might be perceived metaphorically depends on knowledge of a particular understanding of God and His relation to human thinking. I will make interpretative assumptions like, that reason, the mental faculty, is being personified as God's viceroy, an entity that God has granted to the human mind an analogous to an representative of him, implying reason ought to be allied to godliness, but otherwise can be disarmed and restrained by opposed forces.** So this brings us to the question of the meaning of a core selection (of words), "me should defend". It appears, based on this particular interpretation, that it may have a natural meaning. It is possible to transform it, via displacement, into "should defend me", allowing us to see it as an out-of-order sequence, which seems to satisfy our drive for meaning. However, regardless of this, there may also be the temptation to, especially initially, to interpret it as "[I] should defend", a change from objective to subjective case, resulting in a meaning of something like, that the subject ought to be guarded against the disarmament of Reason, but fails to do this. This interpretation also is more ambiguous with regards to time, as Reason is already implied to be captiv'd.



Between these two interpretations (or more, granting that there are many more subtleties that I have left without explicating that could offer more material for interpretation) there lies a special possibility for combining them, or at least holding each of them in mind at the same time and feeling them work together. My experience is that particular kind of interpretative experience only became possible after developing SGP, and this is significant. It is significant here because it seemingly allows a sense of meaning to naturally and fluidly arise out of the text from ambiguously ordered poetic language that otherwise seems to not occur, and this makes out-of-order interpretations not only richer in what might be considered 'the long run' (in the sense of a reader who has spent ample time reflecting on the given poem), but also makes it significantly more apparent immediately that there is something there to these strange twisted expressions and that some real sense can be gained from them, sense that would daunt the reader not  equipped with a vitalistic mind, who at best would have to recourse to understanding any such meaning analytically. The way multiple meanings can so readily and naturally come to mind, in particular in out-of-order expressions, is something distinct about the vitalistically equipped mind that I feel biased it towards being able to make sense out of such poetic licence. This is why I feel there is a particular connection between that I have termed 'multiplicity' and 'out-of-order expressions' which seems to explain why the SEC adds so much to the comprehension of uniquely poetic grammatical constructs.


[FILL IN THIS SECTION]



-This may be used to suggest that, while displacement phenomena alone may not necessarily require the SEC, combinations of displacement and multiplicity can create complexes of meaning that are not going to be interpreted in the same way without it.


-Strength and nobility of displacement of "should defend me" as opposed to "me [I] should defend"



\subsubsection{}



*This shouldn't be surprising in the cases explored here given how the poetic culture that arose in Europe was directly influenced by the conventions of Greek and Latin poetry. The poetry of other cultures seems to bear marks of structure that are similar in kind, though distinct from the specific forms discussed here. But discussion of them is outside the scope of this article.



**This leaves out the issue of Reason proving "weak or untrue" possibly independently of its capture, possibly implying a deeper fundamental weakness of reason itself. In the case of the interpretation "[I] should defend", this might have the further implication that realization of this very weakness may have caused the subject to see holding fast with reason to be not worthwhile.



\section{SPLIT ARTICLE HERE}



"It has often been asked, What is Poetry? And many and various are the answers which have been returned. The vulgarest of all – one with which no person possessed of the faculties to which Poetry addresses itself can ever have been satisfied – is that which confounds poetry with metrical composition: yet to this wretched mockery of a definition, many have been led back, by the failure of all their attempts to find any other that would distinguish what they have been accustomed to call poetry, from much which they have known only under other names." (What is Poetry - J.S. Mill)



Now we come to a certain issue, what might be called a poetic demarcation problem. That is, is there any way to separate 'poetry' from 'non-poetry' so that we can take a given text or utterance and figure out, in general, whether it is 'poetic' or not. It is a problem which a number of people have attempted to give a solution to [2], and yet it is perfectly reasonable to expect that it will have no definite answer given its nature. This issue naturally requires clarification though. We initially have no concrete definition of poetry, and won't unless this particular problem itself is effectively solved. It doesn't really make sense to try to separate a given thing from what it is apparently not without a good method of doing so, especially if the reason for doing this is that it seems to be unclear what the thing is in the first place or how it would even be possible to clearly define it. In which case, why attempt to put serious effort into tackling the problem in the first place, as opposed to accepting that the very concept and how it is constructed may be the result of 'language games'[3] that don't seem to have a clear core?



But maybe there could be a reason to attempt this anyway. Maybe, there could be (there seems to be) some essence, something that seems barely describable, that feels impossible to adequately project into explicit terms, but that which what are commonly called poetic works do seem to have in common. Again, a meaningful description of this essence may be difficult to impossible to state, except via the experience of variations of whatever it is in actual poetry. And even if we could describe 'it', it's not clear what this would achieve, or if my 'it' is the same as yours. In which case, it would be a good idea to narrow the scope a bit.



So are there certain aspects of poetry that are in principle distinguishable from things that lack those aspects? To this I say—from my experience—yes. There absolutely do seem to be particular aspects of poetic experience which should be distinguished from others, namely those that specifically can only be experienced by those with a particular modal faculty (the SEC in this case). To make it clear what I am saying, I am not at all trying to give anything close to an ultimate definition of poetry, there are too many different things, and different aspects of works, that can validly be termed poetic that even attempting such a thing is arguably uninteresting, and ultimately pointless (imagine how useful and illuminating some conception that said, for instance, that a nursery rhyme and a dialog of Plato's were both poetry, would be). But, acknowledging that, I claim that there are particular aspects of the experience of poetry that can absolutely be distinguished in kind from other things, and this is possible by specifically acknowledging that there exist particular kinds of mental experiences and constructs that arise because of and which are dependent on the SEC, and those which are not. This allows a meaningful principle from which such a demarcation can be made. But even then, actually determining what specific works have meaningful aspects to offer that only appear to those with this faculty isn't necessarily so easy. If you aren't able to personally experience a given work both with and without SGP, and perceive things in it you couldn't before that accord with the descriptions I have given (as I have), then the next best recourse is to get a number of people, some of whom possess SGP, and some of whom lack it (which itself also needs to be determined in the first place, possibly via the subjection of individuals to tests), and see if there are aspects that those who possess it can perceive that those who lack it consistently fail to wrap their heads around. For particular metric structures, this shouldn't be out of the realm of feasibility. But for vital metaphor, I suspect that, while not completely impossible, this would be much more difficult, as understanding how one's own mind comprehends metaphor and how this might differ from others in character is no easy task without your own personal experience in memory for comparison. Furthermore, given that our experience of all phenomena is naturally subjective and varying due to individual differences*, to the extent that the possession or non-possession of VI may have subtly different effects on a given person, there is is likely no way to make these kinds of distinctions absolutely.



This is why I emphasize in principal distinguishability, the essential idea that the experience of those with a given modal faculty should differ in predictable ways overall, and that we can look at the difference in judgement of these people and acknowledge how this might occur in order to distinguish works that might be considered high poetry[] in general from works which not appeal to these particular imaginative faculties. As such, we could say things like, that while the particular excerpts form Frankenstein [link this] that were brought up previously were not written with any poetic structure in terms of formal meter or grammar, they do seem to make use of vital metaphor, which demands the ability to imagine metaphoric parallels in a particular way that is unique to the SEC, and as such can be classified as being high or eidosic writing by definition [DEFINE THESE] (which doesn't necessarily mean that the writing is of high quality, even if that is the case here, just that it makes use of this faculty).



[USE QUOTE FROM Attridge 1988: 1, from AMT]



\section{}



The idea of particular kinds of juxtaposition being in some sense the core of poetry as a categorical tool is not an original idea. In particular Roman Jakobson attempted to explore how poetry might be seen in this way and how a theoretical approach could be developed around this. Roman Jakobson was a major figure in linguistics who sought an understanding of poetry through a linguistic lens. He was a strong advocate of poetics being a meaningful sub-branch of linguistics, worthy of study and examination in its own right as any other aspect of language and its use ideally ought to be. This general approach hasn't been as popular as I feel it could have been up until relatively recently, and his work is less used outside linguistics itself—many other departments of humanities seem to be somewhat uncomfortable with his approach and the willingness to look deeply as the possibly technical aspects of a field that is rightly regarded as highly creative and open to interpretation. However, given I have serious insight on these topics to offer myself I believe that this overall attitude deserves a second look.



He chose to identify a single phenomena as the defining feature of poetry, namely, that of parallelism, which in his writing was a generalization of multiple different aspects. The two major aspects that he exemplified under this banner were those of meaning and line...



-Comment on experiencing unity of the poem as a unique factor of SEC and how this factors in to .



SHOW HOW POETRY/NON-POETRY CAN BE IN PRINCIPLE SEPARABLE AND HOW THIS ALLOWS, IN PART, A POTENTIAL SOLUTION TO THE POETIC DEMARCATION PROBLEM. USE ROMAN JAKOBSON'S IDEAS TO DO THIS. 



\subsection{}



Ultimately, the deviance of poetic language seems to characteristically consist somewhat[consist] in the relaxation of conformation to the word orderings otherwise constrained through specific dialects. This doesn't necessarily mean that we can simply use the possibility of occurrence of certain constructs of ordering in poetry in order to reason about the possibility of them (theoretically) occurring or not occurring in a (potentially hypothetical) dialect, but it does seem to be a useful way of thinking about the potential deviation of poetic grammar, and nothing that I'm aware of exists that would explicitly refute this. Thoms comments on this in The Minimalist Theory:



Note that this puts to one side a number of interesting questions

that may have been taken into the scope of the study; for example, how do

speakers of a dialect A read and interpret structures from another dialect B

that would be ungrammatical in A? It is possible that the same mechanisms

are involved for the interpretation of deviant structures in poetry as for the

interpretation of structures that are deviant relative to one’s own dialect, and

thus we might expect that the theory proposed here may be able to extend

its empirical remit to cover such cases, thus creating a general theory for the

interpretation of ungrammatical sentences. This is a significant claim, however,

one that I cannot evaluate in the space afforded by this dissertation.



This may in some sense be true. SGP seems to allow the mind significantly greater ease in comprehending greater varieties of linguistic transformations, even when such orders aren't heard conventionally. On the whole, what we have discussed so far suggests that, even if it isn't necessarily the case that displacement phenomena are themselves impossible to understand for SGPNs, SGP allows richer meaning complexes that can arise from the interaction of alternate word orders.  [FIX THIS]



\section{}



I only encountered Josef Stern's account of the metaphysics of metaphor well after writing the bulk of this article, but having actually read it now I realized how inappropriate it would be to not at least mention it here. I feel that bringing it up is this context is important, but that it really deserves to be given treatment by itself given what now I understand what it actually achieves. [BUILD UP THE REALIZATION THAT HE IN FACT PROVIDES AN ENTIRELY SUFFICIENT INDEPENDENT ACCOUNT OF VITAL METAPHOR]

...seeing that is not seeing and sight beyond sight... Eyesight requires us to perceive the orderedness of the visual plane, and space is calibrated to feelings deriving from the senses as a whole, and in turn calibrates their sense of relative location reciprocally; all of these are built on the biases that necessitate them. [REFERENCE OTHER ARTICLES EXPLAINING THESE THINGS] Vitalistic Imagination is, a way of knitting associations together, which is constructed over and through its own set of biases.





QUOTE (from Metaphor in Context):



1. What kind of knowledge, or ability, enables a speaker-hearer to interpret a metaphor? Is it part of one's general knowledge of language, a species of one's semantic knowledge, the same competence that underlies one's ability to interpret nonmetaphorical, "literal'' language? Or is it, in whole or part, extralinguistic? And in the latter case, is it a yet-to-be-identified power that lies beyond the ordinary speaker's repertoire—a kind of genius, as Aristotle and perhaps Kant hinted? Or is it simply one among the many ordinary (though no less remarkable) abilities we all possess to use ordinary language in an indefinite number of ways?


Hamlet's line (russet mantle) is, as I have suggested earlier, only minimally dependent on VI at best; rather it is more like a picturesque suggestion for the more general imagination. It might also be, in a sense, more 'imagist' in quality, in that it seems to suggest the imagination of more literal images over the vitalistic.



In a sense, I want readers to be able to read through my works in the hope that many of them will feel like I did on reading Stern's account of metaphor, that I am describing clearly aspects of their own experience that they perceived themselves but hadn't been able to elucidate in exactly the same way. And yet I know that many people will find many aspects of my work to be impenetrable. Sometimes this will be because the things being discussed are difficult to describe, but there will also be those who simply lack any experience or memory of any experiences like I am describing and so will naturally not be able to relate to them. If this happens, while I implore the reader to think hard about exactly what I am saying and how I state it, it may also be as important to accept that much of what I am describing will not be intelligible until they have themselves experienced the phenomena I discuss, which may not be a trivial undertaking. The reader may be confronted by their own weaknesses, and must be prepared to accept them if, in grappling with the work, they find their own mind 'coming up short'. The flipside of this is that, if this does occur, I can provide a certain amount of assurance against despair, as I know that these limitations can be overcome by those willing to do what is required.



\section{}



-Nigel Fabb



Genre should not be elevated to anything more than a vessel through which a poet can focus their powers.



\section{}



-Address apparent paradox that literary language is somehow both "elitist" with the claims, by artists (including Wordworth) that they are actually writing for the common man. At least, the artists seem to think that they aren't being deliberately being elitist, but in writing for an audience that (largely) excludes those who lack certain faculties, without necessarily knowing this, they are butting a burden of effort on the (presumably) average person beyond what they realize.



\section{}



*Though in principle an identical person should have identical thought to another unless thought is influenced differently by the non-physical.



[SEC] There are a set of diagrams that I have created in order to illustrate the SEC and how it and the various other faculties and their 'components' relate to each other. This will be presented in a forthcoming post. I use Eidosic as a term to invoke a sense of Eidos or form, yet at the same time to avoid the existing connotations of the term Edetic, which I want to keep it completely separate from, as Eidetic already has existing uses in cognition and philosophy.



url{https://www.britannica.com/topic/eidetic-reduction}

url{https://en.wikipedia.org/wiki/Eidetic_memory}





[MOVEMENT GRAMMAR] The issue of to what extent the modeling of a phrase as a transformation of another—for the sake of making sense of the possibility of the generalization of certain constructs of grammar while still retaining meaningful restrictions on them—is necessary to allow such explanation, is an issue that needs to be understood in sufficient detail in order to see why it may be argued as to why a given phrase may not be "stand alone" (in the sense of the comprehension of its structure being independent of the comprehension of hypothetical transformations) in any given case. So, to make things clear, while I caution against overly presumptive attempts to explain things in terms of transformation grammars, I also don't deny the potential need for such grammatical reasoning.



url{http://lingua.amu.edu.pl/Lingua_17/lin-4.pdf}



[HEGEL] G.W.F Hegel also makes a fundamental distinction between poetry and normal prose, assigning special significance to poetic expression that is not an attribute of prose language or reducible to its meaning.



[J.S. Mill] "It has often been asked, What is Poetry? And many and various are the answers which have been returned. The vulgarest of all – one with which no person possessed of the faculties to which Poetry addresses itself can ever have been satisfied – is that which confounds poetry with metrical composition: yet to this wretched mockery of a definition, many have been led back, by the failure of all their attempts to find any other that would distinguish what they have been accustomed to call poetry, from much which they have known only under other names." (What is Poetry - J.S. Mill)



[STRUCTURE] This is not at all to discount the concept of transformations on a purely structural level; as a theoretical tool the idea of modeling sentences as transformations that modify some other fundamental sentence structure (how or whether or not it exists in some specific form in the mind), is likely highly useful. [ANALYSIS] What I am cautioning against is the idea that a given structure that may in some way derive from a 'more fundamental' structure will have the same fundamental meaning as that structure, as opposed to having nuances that are generally interpreted differently. This this regard my views are closer to those expressed by Owen Barfield, in that at the very least I take it that the impression that a given expression leaves on the reader/listener is meaningfully effected by the form of the "surface utterance", which can't simply be assumed to be equivalent to whatever they may be considered a transformation of. I believe that this same principal applies to "canonical sequences" and the transformation of themes in music as well.



[QUOTE T.S. ELIOT]

"The music of a word is, so to speak, at a point of intersection: it arises from its relation first to the words immediately preceding and following it, and indefinitely to the rest of its context; and from another relation, that of its immediate meaning in that context to all other meanings which it has had in other contexts, to its greater or less wealth of association . . . My purpose here is to insist that a "musical poem" is a poem which has a musical pattern of sound and a musical pattern of the secondary meanings of the words which compose it, and that these two patterns are indissoluble and one • . I believe that the properties in which music concerns the poet most nearly, are the sense of rhythm and the sense of structure" (1975, p. 113)



[ANALYSIS] There are certain schools of linguistics that would deny that the fundamental capacity for language arises out of a specific, pre-set biases towards the learning of certain classes of structures, instead choosing to propose instead that all structure can arise out of more basic or general statistical models, and as such that the nature of what might be called "the language faculty" is almost entirely emergent based on individual experience, without the sharp divides assumed by generative theorists. This is an interesting issue. However, while there may be a temptation to throw away the assumptions that lead to such formal analysis, the fact that humans (and humans alone) tend to generalize grammar in particular ways, and not others, such that the forms the emerge seem to (largely*) conform to particular classes of computational complexity, is something that at least implies that exploring such generative analytical approaches is at least reasonable. Maybe the generative language faculty is 'real' in that it does bias towards these analytical structures and is biased to 'come forth' as a human response to certain kinds of input, while also being 'emergent', in the sense of arising out of a need, and being capable of failing to arise in specific ways. My experience of activating EVP suggests that something like this may be the case. This is why, despite the way that poetry allows such actual freedom of structure and imagination, I am choosing not to simply abandon generative approaches in analyzing it, and make use of them to understand the relation between ordinary language and poetry. Yet, I still stress that analytic reductions, attempts to explain poetry in terms of set-derivative rules, is necessary limiting and will fail to capture how structure enables actual expression (poetic or otherwise) in its fullness. But by making use of such analysis, I can show how potentially distant poetic expressions are from ordinary expression in terms of the transformations that would be needed in order to make them interpretable as grammatical sentences with predicative meaning if we assumed they were to be understood in that way. So analytic and generative tools are things I content are highly useful when applied appropriately, but should not be treated as rigid designations of all that is possible in language and expression.



https://munin.uit.no/bitstream/handle/10037/10310/article.pdf



\subsubsection{}



*The exceptions to this are interesting in their own right, and will be the subject of further writings.



\subsubsection{}



[SELECTION] However, the approach of selecting a corpus of texts pre-divided into the 'poetic' and 'non-poetic' at the digression of the writer is not one I wish to primarily take with regard to using them to distinguish aspects of writing into different classes, and isn't how I want to make the distinctions I want to draw attention to, as this "process of selection" can allow someone to choose texts that validate their own theories, even if a different selection possibly wouldn't. I actually select the texts I make use of here in such a deliberate manner though, with the aim of using them to illustrate the points I wish to make, but in principle, with regard to the processes used to distinguish, it should be possible to look at the nature of poetic and literary language in terms of how communities of interpreters read and experience them, and to also examine how the makeup of those communities—in terms of their possession or non-possession of specific modal faculties—influences their interpretations. This, though difficult in practice, allows a broader means of making such distinctions so as to potentially go further beyond the biases of an individual interpreter. This is not to reduce the wealth and richness of a given person's personal interpretation in favor of an abstract view of a 'communal interpretation' though, but rather to acknowledge the influence of the individual perception on literary communities, which should in principle allow someone to take any text and examine how people with different faculties, cultural experiences and perceptual abilities experience that given text, and see how these factors impact the perception and reception of it.



Thoms justified his choices in part with the claim that:

The problem is that the categorization of some sets of examples as “artificial”

and others as “natural” is almost entirely arbitrary and based on extralinguistic

factors. Such a partitioning is both theoretically and methodologically unsound,

and I believe it should be avoided.



But this disregards the nature of the decision to choose texts he considers 'poetically divergent', yet at the same time restricting his text to those which are explicitly transformations of a series of 'ordinary' sentences with definable conceptual/propositional meaning. I also offer different methods of potentially distinguishing between 'natural' and 'artificial' poetically grammatical constructs that are potentially methodologically viable, through in principle distinguishability.



This footnote is also mentioned in the discussion of the GTTM and its relation to the work. It is a counterclaim to what is stated in Nigel Fabb's "Is literary language a development of ordinary language?":


Note that there are important differences between Lerdahl and Jackendoff’s work on the

one hand and the work on poetic language on the other, since Lerdahl and Jackendoff do not set

out to study music as a special variant of some other basic component of cognition. Effectively

their object of study is the standard language or prose equivalent in musical cognition, rather

than some deviant form like literary language.



But through my own framework, I explicitly deny that this is the case. Rather I propose that, in requiring SGP in order for grouping structure to be intelligible, the GTTM is a theory that covers exactly the musical equivalent of 'conventional' poetry, as opposed to either ordinary speech or the avant garde. And my theory as a whole specifically deals with the structures and phenomena that might be associated with specific faculties, and as such I choose works based around observations of which works are actually transformed in perception via those faculties, and don't use those which I personally have failed to find more intelligible after their activation. However, I will potentially admit the input of others in order to better understand what kinds of works and forms may be enhanced by the possession of each faculty and the development of their use in other individuals.



Thoms also make this observation about theories that make use of 'pragmatic' (i.e. non-truth-propositional) meaning:



However, such theories suffers from a number

of theoretical and empirical problems. Perhaps the biggest theoretical problem

is that the theory is almost entirely unconstrained and, as a result, largely

uninteresting from the perspective adopted in the introduction to this thesis.

The introduction set out the goal to develop a theory of poetic language that

can explain the kinds of poetic language we find and the kinds that we do not

find, and as such the theory must be strongly predictive and constrained. Yet it

is difficult to envisage constraints on the very general operations of pragmatics

that will adequately constrain the theory, and in the absence of such constraints

we would be unable to control the power of the pragmatic component.


Now, for the purposes of that particular thesis, this makes sense. Given that his goal is to explain why it might be that the particular structures found in poetry happen to be, while other structures (so far)

are not generally observed, in the framework of generative linguistics, it makes sense that he would restrict his example works to those he has the theoretical tools to deal with. However, this is not my goal, as I want to explore the possible effect possessing or not possessing particular mental faculties has on the experience of poetry, along with the nature of the significant structures this seems to give rise to. As such, my theory naturally deals with phenomena outside of the scope of The Minimalist Theory, as it has the motivation and means to. Most importantly, I have no reason to pay as much attention to the non-existence of certain forms in corpora of poetic language, as the distinctions between experiences based on the activation or non-activation of a faculty are sufficient for my purposes.



[3 LANGUAGE GAMES] In Wittgenstein's Philosophical Investigations, the idea of a 'language game' is introduced, which might be considered to be a context that is experienced such that within it the meaning of particular given utterances emerge. As a preliminary example of the kind of thing he is talking about, Wittgenstein provides the example of someone, such as a foreman, shouting "slab", which, in the particular work environment in which it it is uttered, has come to be recognized as a particular signal to a worker that a particular type of concrete slab should be delivered to a particular location. An important consequence of this that none of this or its implied meaning needs to be something like a 'noun' connected to any kind of wider language or grammar, and this meaning takes upon its role as a signal, which is what it means, due to it being recognized by a person as a convention within a particular lived role, and that this is intuitively possible even in the absence of other linguistic conventions (Would it be meaningful to consider, in this particular context, whether the term "slab" is used as a noun or a verb? Thought I must point out that if we cannot easily classify it as either in this situation, this doesn't necessarily mean that this ambiguity would remain in the case that we placed it in a larger linguistic context by appending the word "This..." or the phrase "This is a..." in front of it. It may be that placing it within such contexts forces the sense of nounness or verbness onto  the term via that nature of the natural biases of grammar). Going further with this, a given utterance can pick up various meanings, possibly related in various ways, which might have different nuances in different contexts, but ultimately the meanings are seemingly 'grounded' in how they relate to our experience, rather than referring to merely 'pre-given' definitions independent of experience. The word 'game' is used to illustrate a possible falling out of this, namely, that there are things, each of which may be called games, which, while they do share certain aspects in common with other things called games, nevertheless may have no specific thing in common with each other. This kind of broad relation is termed a 'family resemblance'.

So, with respect to poetry, in what sense might the term, poetry, in all the context in which it is used, be a guide to the various things that poetry labels and how they might be connected? Poetry is not cognized as a 'simple' act, but rather seems to be connected to various things that, while particular pairs of examples of these things may seem to be mutually related to each other, other pairs may end up being so distant as to have no shared aspect in common at all apart from the term they are labeled with. The sense of "what poetry is" hence lacks the sense of simplicity and obviousness that the act of moving a slab and the connection of this with an utterance would seem to have.

So we might consider this issue to be 'settled', and see 'poetry' as something like 'game', which consists of a common definition with seemingly no 'core'. But I think we can gain something by not being too hasty in this pronouncement. There is nothing preventing us from looking at specific, particular aspects of what is called poetry and exploring how we might see them as distinct characteristics of something meaningful, as so I take this route here to see what might result from it.



[consist] I originally wrote this sentence as "consist characteristically" then considered how poorly "somewhat" would fit in when used afterwards. I just wanted to note this due to its relevance to the topic at hand.



[1] Poetic language: a Minimalist theory,  Gary Stewart Thoms, 2010

[2] A chief, important figure in the development of theoretical, linguistic approaches to poetics was Roman Jakobson.

[3] Reference Wittgenstein's Philosophical Investigations, in which the concept of a 'language game', where the definitions of things may be recognized to consist in nebulous social conventions.

\end{document}