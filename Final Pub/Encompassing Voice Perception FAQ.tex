\documentclass[]{article}
\usepackage[breaklinks]{hyperref}

%opening
\title{Encompassing Voice Perception FAQ}
\author{Andrew Van Caem}
\date{4th November 2016}

\begin{document}

\maketitle

\textbf{Q: What is EVP?}\\

\textbf{A:} Encompassing Voice Perception (EVP) is the mental stage of development that needs to be reached in order to hear "tonally centered" music. This, to all non musicians out there, is any form of music which relies on tension, and the release of that tension through the completion of melodic phrases. If you're not sure if you listen to "tonally tensioned" music (as I will define it), then chances are you only listen to milder forms of it at best, as heavily tonally centered music will likely sound incredibly dissatisfying to listen to if you lack encompassing voice perception. Tonally centered music is 'IN' a key. This means that the whole piece of music is inside a tension that demands resolution by the playing of a specific note or class of notes at the appropriate time (which is traditionally termed a cadence). I will hence use "tonally tensioned" to describe music that is specifically "inside of a key tension".\\

This ability to hear that music can be "in a key tension" is also exactly what allows a given person to experience a piece of music as having musical 'themes', which are classes of melodies that can be perceived as transforming into one another over the course of a piece. This is something that notably does not occur without the capacities enabled by EVP, and so there will be no sense of musical themes transforming or developing without it. Instead, any music based around complex transformations of theme (such as a fugue \url{https://youtu.be/bOWi8tOf5FA}) will be perceived as having pointless, arbitrary structure that serves no purpose, has no perceivable guiding principles and seemingly makes no sense.\\

\textbf{Q: Why is it called Encompassing Voice Perception in the first place?}\\

\textbf{A:} Compared to non EVP sound perception, it is as if the whole world is speaking to you through each and every sound. If you have EVP and do not feels this is the case, I'd argue that you are so used to perceiving the world in this way that you don't think it is anything unusual. And essentially, it isn't. Many people go through most of their lives with EVP active, but to consciously experience the transition from not having EVP to having it is quite unusual, and this quality of it actively stands out. The key difference, I believe, is that the language centers of the brain actively process sound in general in people who have EVP, while only selectively in those without it, though I can't be sure until there is experimental data backing this up.\\

\textbf{Q: Is everyone capable of EVP?}\\

At a given point in time, no not everyone necessarily has it, and they are less likely to have it the younger and less experienced they are, but in the longer run it should be possible for most anyone with normal hearing to develop it, and this is the core of the claim that I am making.\\

\textbf{Q: Are these people who lack EVP tone deaf?}\\

\textbf{A:} No, not at all. In fact, lacking EVP is no barrier at all to listening to music that isn't based heavily around constructing and developing tonal tension, which can be rich in variety and intensity of emotion regardless.\\

\textbf{Q: Hey, I'm a musician or otherwise have some knowledge of music theory. Are you claiming that there are people who simply cannot hear or understand 'tonal music', at all?}\\

\textbf{A:} In a very real sense, yes, and not only am I making this case, but I'm claiming that this is likely to be completely normal to lack EVP, to the point that in a given cultural environment, the number of people without EVP may outnumber those with it. There still may be people who do not have EVP, but are able to appreciate "tonally centered" music anyway, though I have no idea to what extent this may occur. I can't rule out the possibility that some people may interpret only certain styles of music using the language facilities of the brain due to exposure to the sounds of particular instruments the womb or early childhood, which may lead to them interpreting those instruments in the same manner and using the same brain areas they do with voices, though this is complete conjecture in my part with no instance backing it up. The basic gist of EVP though is that it implies an expansion of what the mind interprets as language to all possible sound (as opposed to this being limited to that of what is recognized as the human voice).\\

\textbf{Q:} You say "in a given cultural environment"? What do you mean by that and are you implying that things could be different?\\

\textbf{A:} I suspect that the culture surrounding a person and the environment they grow up in is a huge influence on whether or not someone possesses EVP, but that different cultures and eras may have different proportions of their population equipped with EVP.\\

\textbf{Q:} How do I know if I have EVP?\\

\textbf{A:} I do not currently know the best methods of determining this, as even someone with EVP will have to pay some attention in order for structural potential it allows to be apparent. There are particular pieces of music that one can listen to that may allow you to gauge whether or not you have EVP, such as the Chopin op 27 piece I recommend in this other article, but I'm willing to work on discovering other ways of determining this and describing specifically how the nature of musical themes and theme transformation, (which should not occur without EVP) along with tonal resolution differ from the ordinary way people without EVP hear regular melody. What's important to understand is that lacking EVP means that no matter how much effort you put in you will not be able to perceive the elements unique to EVP that I describe here, at least until you trigger the mental transformation that enables EVP (described in this post).\\

\textbf{Q: If I don't have EVP, is there some way that I can still appreciate 'tonal' music? What is this transformation you speak of?}\\

\textbf{A:} No, and this is my second major claim, people who currently lack EVP may be able to transform their mental perception of sound so that they do perceive Encompassing Voice if they put in the effort to do so. I must stress though that this involves a sudden and significant transformation of the way you perceive sound and may not be easy to trigger. I suspect that children who study music seriously may undergo this transformation quite frequently though.\\

\textbf{Q: Is non "tonally centered" music then 'atonal'?}\\

\textbf{A:} No necessarily. Much pop music is NOT necessarily "tonally tensioned" or "inside of a key". It often is not structured in such a way that demands a cadence, and so can fade out or repeat parts of its structure indefinitely. This is different from how 'tonality' is often defined in traditional tonal music. The fact that the music is often based around particular scales that are derived from a certain key is different from, but related to, the nature of the tension perception it demands. You certainly don't have to be Schönberg to write non "tonally tensioned" music.\\

\end{document}